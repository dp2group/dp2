\section{Merging}

When two users merge, the user with the higher \emph{uid} is selected to be
the merger. All merging is done on a pairwise basis. If n users are online 
and another comes online, the user with the highest \emph{uid} out of the currently
online group is chosen to do a pairwise merge with the newcomer.

When this happens, the user with the lower \emph{uid} (user A) looks up the last merge point
with the higher \emph{uid} (user B), and sends everything in his log past that point to
A. User B also scans back in the log to the last merge with A and copies this to a new point 
in memory. Using the two differing log parts, User B then resolves conflicts as described below.

When the conflics are resolved, User B appends all of the changes from User A to the end of his log
and applies them to his working copy. He sends the portion of his log after the previous merge with
A to A, and appends a "MERGE A" to the end of his log.

When User A receives the new log from B, he rolls back all of his changes so that his document appears
as it did at the last merge with B. He then applies the new changes from B and replaces his portion
of the log since the last merge with the resolved log.

\subsection{Conflict Definitiion}
With the current log representation, there are a number of conflicts that need to be resolved manually
by the merging user. These conflicts are:

\begin{enumerate}[1)]
\item The two users change the same sentence in different ways. A deletion is considered a change.

\item The two users move a sentence to different locations.

\item The two users move a paragraph to different locations.

\item One user deletes a paragraph, another edits its contents (changes a sentence).

\item One user deletes a sentence, another changes it.

\end{enumerate}

These are not considered conflicts, and will be ignored:

\begin{enumerate}[A)]
\item The two users change a sentence to the same thing.

\item The two users move a sentence to the same location.

\item The two users move a paragraph to the same location.

\item The two users delete the same sentence/paragraph

\item One user edits a paragraph, one moves it.
\end{enumerate}

\subsection{Conflict Resolution Algorithm}
This algorithm identifies conflicts and presents them to the user for resolution.
It takes as arguments two log file parts from different users.

The algorithm works by traversing the two log parts backwards (from most recent change
to oldest change) and determining if the most recent changes to each paragraph and sentence
conflict between the two log parts. This is accomplished by first traversing backwards
through the log of the merging user and building up two tables. These tables look like this:

%%-------------------------Paragraph Table-----------------------------------%%
\begin{table}[h!]
\begin{center}
 \begin{tabular} {|l|l|l|l|l|}
  \hline
   \emph{pid} & Modified & Deleted & Moved & New Location \\
  \hline \hline
    421 & 1 & 1 & 0 & 0 \\
   \hline
    512 & 0 & 0 & 1 & 421 \\
   \hline
    21 & 1 & 0 & 0 & 0 \\
   \hline
    893 & 1 & 0 & 0 & 0 \\
   \hline
 \end{tabular}
\end{center}
\caption{Table tracking changes to paragraphs}
\label{table:paragraph_table}
\end{table}
%%-----------------------------------------------------------------------------%%

Table \ref{table:paragraph_table} (above) shows the table used for tracking paragraph
changes. Table \ref{table:sentence_table} (below) shows the table used for tracking
sentence changes. Both are built dynamically, there are not entries for each paragraph
and sentence when they are created. 

%%-------------------------Sentence Table-----------------------------------%%
\begin{table}[h!]
\begin{center}
 \begin{tabular} {|l|l|l|l|l|l|}
  \hline
   (\emph{pid}, \emph{sid}) & Modified & New Content & Deleted & Moved & New Location \\
  \hline \hline
    (421, 75) & 1 & "New Sentence" & 0 & 0 & 0 \\
   \hline
    (512, 32) & 1 & NULL & 1 & 0 & 0 \\
   \hline
    (21, 88) & 0 & NULL & 1 & 0 & 0 \\
   \hline
    (21, 99) & 0 & NULL & 0 & 1 & 34 \\
   \hline
 \end{tabular}
\end{center}
\caption{Table tracking changes to sentences}
\label{table:sentence_table}
\end{table}
%%-----------------------------------------------------------------------------%%

These tables track certain characteristics useful in conflict resolution. The algorithm
then traverses the imported log part in a backwards manner, following these rules:

\begin{enumerate}[1)]

\item When encountering a DELETE\_P, if the modified bit is set in the paragraph table for this paragraph,
      there is a conflict and the user is prompted. If the user keeps the deletion, the
      statement remains in the log, otherwise it is removed.

\item When encountering a DELETE\_S, if the modified bit is set in the sentence table for this sentence,
      there is a conflict and the user is prompted. If the user keeps the deletion, the
      statement remains in the log, otherwise it is removed.

\item When encountering a CHANGE\_S, if the most recent change to a sentence changes it to different text
      than is recorded in the table, prompt the user to fix the conflict. If they choose the local sentence,
      or the most recent change is the same as what as in the table, delete this and all previous changes
      to this sentence in the foreign log. If they choose the foreign sentence, delete changes in the 
      local log.

\item When encountering a MOVE\_P, if the moved bit is set in the paragraph table for this paragraph,
      check if the targets are the same. If they are different, prompt the
      user to resolve the conflict. If he chooses the local version, or the targets were the same, 
      delete all moves to this paragraph
      from the foreign log part. If he chooses the foreign version, delete all moves to this paragraph
      from the local log part.

\item When encountering a MOVE\_S, if the moved bit is set in the sentence table for this sentence,
      check if the targets are the same. If they are different, prompt the
      user to resolve the conflict. If he chooses the local version, or the targets were the same, 
      delete all moves to this sentence 
      from the foreign log part. If he chooses the foreign version, delete all moves to this sentence 
      from the local log part.

\item Keep all MERGE and CREATE statements in the log.
\end{enumerate}

\subsection{No Conflict Example}
User 1 and 2 are editing a document while offline. User 1 edits paragraph 5, sentence 1. User 2 edits
paragraph 9, sentence 3. Both make the same edit to paragraph 1, sentence 3. User 2 moves paragraph 5
to before paragraph 1. Before merging, their logs look like this:

%%-------------------------Ex 1 Before-----------------------------------%%
\begin{table}[h!]
\begin{center}
 \begin{tabular} {|p{7cm}|p{7cm}|}
  \hline
   User 1 Log & User 2 Log \\
  \hline \hline
   MERGE 2 & MERGE 1 \\
   CHANGE\_S 5, 1, "Old", "New" & CHANGE\_S 9, 3, "Banjo", "Ukelele" \\
   CHANGE\_S 1, 3, "Dog", "Cat" & CHANGE\_S 1, 3, "Dog", "Cat" \\
           & MOVE\_P 5, 3 \\
  \hline 
 \end{tabular}
\end{center}
\caption{Logs before merge}
\label{table:ex1_before}
\end{table}
%%-----------------------------------------------------------------------------%%

After the merge algorithm, the logs look like this:

%%-------------------------Ex 1 After-----------------------------------%%
\begin{table}[h!]
\begin{center}
 \begin{tabular} {|p{7cm}|p{7cm}|}
  \hline
   User 1 Log & User 2 Log \\
  \hline \hline
   MERGE 2 & MERGE 1 \\
   CHANGE\_S 5, 1, "Old", "New" & CHANGE\_S 5, 1, "Old", "New" \\
   CHANGE\_S 1, 3, "Dog", "Cat" & CHANGE\_S 1, 3, "Dog", "Cat" \\
   CHANGE\_S 9, 3, "Banjo", "Ukelele" &  CHANGE\_S 9, 3, "Banjo", "Ukelele"  \\
   MOVE\_P 5, 3  & MOVE\_P 5, 3 \\
   MERGE 2 & MERGE 1 \\
  \hline 
 \end{tabular}
\end{center}
\caption{Logs after merge}
\label{table:ex1_after}
\end{table}
%%-----------------------------------------------------------------------------%%

In this case, since there are only two users the log before the second set of merges
could be garbage collected.

\subsection{Conflict Example}

User 1 and 2 are editing a document while offline. 
User 1 edits paragraph 5, sentence 1. 
User 2 edits paragraph 9, sentence 3. 
User 2 edits paragraph 9, sentence 7.
Both make different changes to paragraph 1, sentence 3. 
User 2 moves paragraph 5 to before paragraph 1. 
User 1 deletes paragraph 9. 
Before merging, their logs look like this:

%%-------------------------Ex 1 Before-----------------------------------%%
\begin{table}[h!]
\begin{center}
 \begin{tabular} {|p{7cm}|p{7cm}|}
  \hline
   User 1 Log & User 2 Log \\
  \hline \hline
   MERGE 2 & MERGE 1 \\
   CHANGE\_S 5, 1, "Old", "New" & CHANGE\_S 9, 3, "Banjo", "Ukelele" \\
   CHANGE\_S 1, 3, "Dog", "Cat" & CHANGE\_S 1, 3, "Dog", "Kangaroo" \\
   DELETE\_P 9, 12 & MOVE\_P 5, 3 \\
  \hline 
 \end{tabular}
\end{center}
\caption{Logs before conflict resolution and merge}
\label{table:ex1_before}
\end{table}
%%-----------------------------------------------------------------------------%%

In this case, the merge algorithm will detect two conflicts: Users 1 and 2 make different
changes to paragraph 1, sentence 3, and User 1 deletes a paragraph that User 2 changed.

Assuming the user chooses to select User 2's changes to sentence (1, 3), and to accept
User 1's deletion of paragraph 9, the log would then look like this:

%%-------------------------Ex 1 After-----------------------------------%%
\begin{table}[h!]
\begin{center}
 \begin{tabular} {|p{7cm}|p{7cm}|}
  \hline
   User 1 Log & User 2 Log \\
  \hline \hline
   MERGE 2 & MERGE 1 \\
   CHANGE\_S 5, 1, "Old", "New" & CHANGE\_S 5, 1, "Old", "New" \\
   DELETE\_P 9, 12 & DELETE\_P 9, 12 \\
   CHANGE\_S 1, 3, "Dog", "Kangaroo" & CHANGE\_S 1, 3, "Dog", "Kangaroo" \\
   MOVE\_P 5, 3 & MOVE\_P 5, 3 \\  
   MERGE 2 & MERGE 1 \\
  \hline
 \end{tabular}
\end{center}
\caption{Logs after conflict resolution and merge}
\label{table:ex1_after}
\end{table}
%%-----------------------------------------------------------------------------%%

During conflict resolution, the lines 'CHANGE\_S 1, 3, "Dog", "Cat"' and 
'CHANGE\_S 9, 3, "Banjo", "Ukelele"' were removed from the log due to the user's
merge choices.

As there are only two users in the group, this log could also be garbage collected.




