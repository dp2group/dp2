\section{Tracking Changes}

Each user maintains two data strucutres beyond their document
representation: a change log and a table mapping the other users in 
the group to points in the log. 


\subsection{Log}
As each user edits their local copy of the document, their changes
are continually appended to their local log. The table of commands in
Table \ref{table:log_commands} are appended sequentially to the log as the user edits the in-memory
document. In the table, the symbol \emph{pid} refers to a unique paragraph ID, the symbol \emph{sid}
refers to a unique (within the associated paragraph) sentence ID, and the symbol \emph{uid} refers
to a unique user ID.

%%-------------------------Log Command Table-----------------------------------%%
\begin{table}[h!]
\begin{center}
 \begin{tabular} {|c|c|p{8cm}|}
  \hline
  Log Command & Parameters & Description \\
  \hline \hline
  CREATE\_S & \emph{pid}, \emph{sid}, \emph{pos} & Created a new sentence with unique ID \emph{sid} at position \emph{pos} in the paragraph with ID \emph{pid} \\
  \hline
  CREATE\_P & \emph{pid} \emph{pos} & Created a new paragraph with the unique ID \emph{pid} at position \emph{pos} \\
  \hline
  DELETE\_S & \emph{pid}, \emph{sid}, \emph{p\_old} & Deleted the sentence with ID \emph{sid} from position \emph{p\_old} in the paragraph with ID \emph{pid} \\
  \hline
  DELETE\_P & \emph{pid}, \emph{p\_old} & Deleted the paragraph with ID \emph{pid} from position \emph{p\_old} \\
  \hline
  MOVE\_S & \emph{pid}, \emph{sid}, \emph{p\_old}, \emph{p\_new} & Moved the sentence at (\emph{pid}, \emph{sid}) 
                                                                         from position \emph{p\_old} in paragraph \emph{pid} 
                                                                         to position \emph{p\_new} in paragraph \emph{pid} \\
  \hline
  MOVE\_P & \emph{pid}, \emph{p\_old}, \emph{p\_new} & Moved the paragraph with ID \emph{pid} from position \emph{p\_old} 
                                                                                              to position \emph{p\_new} in the document \\ 
  \hline
  CHANGE\_S & \emph{pid}, \emph{sid}, \emph{s\_old} \emph{s\_new} & Changed the sentence at (\emph{pid}, \emph{sid}) from the text \emph{s\_old}
                                                              to the text \emph{s\_new}  \\
  \hline
  MERGE & \emph{uid} & Merged with the user with ID \emph{uid} at this point in the log \\
  \hline
 \end{tabular}
\end{center}
\caption{The log commands used to track document changes}
\label{table:log_commands}
\end{table}
%%-----------------------------------------------------------------------------%%

\subsubsection{Example Log}
This example demonstrates how a log would track a user's changes. Assume user 1 is editing a document
with user 2, and the two have just merged their changes. At this time, there are three paragraphs with IDs 98, 100,
and  273, in that order. Paragraph 273 has two sentences with IDs 104 and 105. 

The user then changes the first sentence of the last paragraph from "I saw a dog" to "I saw a cat". Next, he moves the last
paragraph to the front of the document. Finally, he adds a new paragraph at the end of the document with one sentence:
"This is the conclusion".

After these changes, the document will look like this:

%%-------------------------Example Log-----------------------------------%%
\begin{table}[h!]
\begin{center}
 \begin{tabular} {|l|}
  \hline
   User 1 Log \\
  \hline \hline
   MERGE 2 \\
   CHANGE\_S 273, 104, "I saw a dog", "I saw a cat" \\
   MOVE\_P 273, 2, 0 \\
   CREATE\_P 233, 3 \\
   CREATE\_S 233, 10 \\
   CHANGE\_S 233, 10, "", "This is the conclusion" \\
  \hline
 \end{tabular}
\end{center}
\caption{The log commands used to track document changes}
\label{table:log_commands}
\end{table}
%%-----------------------------------------------------------------------------%%

The IDs in the CREATE\_S and CREATE\_P statements generated by the user to be unique.

\subsection{Checkpoint Table}

Every user maintains a table tracking, for each user, a pointer to the most recent MERGE statement.
This provides for two things: first, when merging with another user, it allows the merger
to quickly determine the last common point between the two. Second, it allows for parts
of the log to be garbage collected when they are no longer needed.

The checkpoint table should be stored as a hash table to allow $O(1)$ lookup times. The below
table simplifies this representation, but shows what information should be included.

%%-------------------------Checkpoint Table-----------------------------------%%
\begin{table}[h!]
\begin{center}
 \begin{tabular} {|c|c|}
  \hline
  User ID & Last Merge \\
  \hline \hline
   324132 & 99 \\
  \hline
   43213 & 100 \\
  \hline
   923138 & 99 \\
  \hline
   173295 & 80 \\
  \hline
 \end{tabular}
\end{center}
\caption{A representation of the data stored in the checkpoint table}
\label{table:checkpoint_table}
\end{table}
%%-----------------------------------------------------------------------------%%

In the example above, there are five users in the group.
Table \ref{table:checkpoint_table} represents the table of one individual user.
In the above example, every line in the log above line 80 can be garbage collected, as
the information is no longer needed. The user has most recently merged with user 43213,
since their merge entry is most recent. The system can also determine if a merge between
two users is necessary by inspecting the log at the appropriate line and scanning forward
to see if there have been any changes.














